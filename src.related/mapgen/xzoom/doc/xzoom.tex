\documentstyle{article}
\title{Description of Program {\bf xzoom}, v1.03}
\author{Gerald I. Evenden}
%\def\baselinestretch{1.5}
\begin{document}
\maketitle
\vspace{1.25in}
\special{psfile='`zcat icon.ps' hoffset=138}
\section{Introduction}
Program {\bf xzoom} is an interactive means for viewing the
vector graphic {\sc mapgen} and {\sc plotgen} overlay files
in the {\bf X11} environment.
In addition to displaying overlays, cursor position and
text data can be digitized and placed into a program {\bf legend}
compatible format to be used for the creation of new overlays.

Use of the program is by mouse selection from a
menu of command buttons on the user's screen.
Some commands will pop up additional screens for entering
information such as file names, working directory path, or
editing a file of a list of overlays to be displayed.
Other commands are related to either display control or digitizing
operations.

Effort has been made to make {\bf xzoom} reasonably ``user friendly''
and requiring as little learning time as possible for basic usage.
But the new user should read at least the section on {\bf xzoom}
execution or receive about 30 minutes of verbal instruction provided that
reasonable preconfiguration and initialization procedures
have been performed.
If the reader is learning to use this program from this manual
it is recommended that it be in conjunction with concurrently
executing {\bf xzoom}.

Installation and configuration of {\bf xzoom} requires some
familiarity with the {\bf X11} system.
Details of the resources file and recommended or suggested
parameters are discussed in later sections of this manual.

\subsection{History}
{\bf Xzoom}'s predecessors date back to the original {\bf zoom}
program designed for older Tektronix 4010 style graphic devices and which
originally performed only basic display operations.
Later revisions added digitizing which was very tedious
because of the frequent lack of a mouse and the difficulty
of menu displays on these types of devices.
Installation of {\sc mapgen} on Sun Microsystems workstations
resulted in program {\bf szoom} that was a trend setter in
terms of features and mechanisms used in {\bf xzoom}.

\section{xzoom Execution}
Execution of {\bf xzoom} depends a great deal upon the user's
particular {\bf X11} environment.
For example, if the user is at server which is not also a client
then {\bf xzoom} will have to be executed via a remote login to
the client machine or possibly a remote shell command.
When the server and client are the same machine execution
will be dependent upon the user's {\bf X11} environment.
Readers is in doubt should consult local personnel familiar
with this system.
\begin{figure*}[h]
\vspace{1.5in}
\special{psfile="`zcat init.ps"}
\caption{Initial screen appearance of {\bf xzoom}.}
\label{init}
\end{figure*}

For descriptive purposes, however, is is assumed that the
user is employing a {\bf csh} window such as {\bf xterm}
and a window manager such as {\bf twm} and the system will act as
both a client and server.
Prior to executing {\bf xzoom} the user should make sure
that appropriate {\bf xzoom} resources are set by executing:
\begin{verbatim}
xrdb -merge ${HOME}/XZoom
\end{verbatim}
In this case, the resources file {\tt XZoom} is considered
to be in the user's root directory.
This operation will usually become part of an active {\bf xzoom} user's
start up scripts or {\tt .Xresouces} file.
The next operation is simply:
\begin{verbatim}
xzoom
\end{verbatim}
What happens next is dependent upon the window manager---{\bf twm}
will place a blinking grid on the screen and expect the user
to position it with the mouse and click button 1.
At this point, {\bf xzoom} should be displayed and ready for operation
as seen in Figure~\ref{init}.

The first thing to note is that {\bf xzoom}'s
display is divided into five vertical levels which are, starting from
the top:
\begin{enumerate}
\item command buttons which control basic operations,
\item instruction or executional reminders for certain operations
and error messages,
\item text editing area for labels and {\bf legend} commands,
\item background, and pen color definitions, and
\item graphic display area.
\end{enumerate}
Most of the following description will be related to the
command button in the first level.

When first executed only a few of the command buttons are enabled
such as {\tt quit} (as shown by their highlighting),
and as the user proceeds through
the proper sequence, appropriate command buttons are enabled.
For example, the graphic display operations {\tt (re)plot},
{\tt zoom}, etc.
are only enable when a {\sc mapgen} or {\sc plotgen} definition
file has been specified with the {\tt def} button and its
associated popup request for a file name has been satisfied.
Selecting any one of the commands is performed by moving the
cursor with the mouse into the command box (which causes its border to expand)
and clicking mouse button 1.
Note that the first button with a number displays {\bf xzoom}'s
version number and is not a command.

The following is a discussion of each of the general command buttons.
\begin{description}
\item[\fbox{\tt RESET}]
This command button is only activated by error conditions or messages
which are displayed in the comments window below.
Once the message has been read, toggle the button to enable
the other commands.
\item[\fbox{\tt quit}]
Terminate the execution of {\bf xzoom}.
\item[\fbox{\tt cwd}]
The user may change working directory with this command button.
A popup window will appear with the current working directory
which may be edited to a new directory.
A ``\verb@~@'' prefix may be employed to denote the use of
the user's {\tt HOME} environment variable as a prefix.
Selecting the {\tt ENTER} button returns to the main
{\bf xzoom} command window.
\item[\fbox{\tt def}]
This command pops up a request for the name of a {\sc mapgen}
or {\sc plotgen} definition file.
The definition file may be changed at any time but a valid
file must be specified by this command
before graphic or digitizing operations are allowed.
Initially, a default name of {\tt def} is specified.
The command button {\tt ENTER} attempts to execute the
directory change and returns to the main commands if successful.
If the directory change is unsucessful, an error message
is issued.
{\tt ABORT} will return to the main commands without further
action.
\item[\fbox{\tt digit}]
Before digitizing operations are permited the {\tt digit}
command must be used and an appropriate file name
entered in the pop up window.
A default name of {\tt digit} is initially specified.
If the selection of a digitizing file is made prior to
the selection of the {\tt def} file the digitizing commands
remain suspended until the later is specified.

Press {\tt ENTER} when the file name is edited to the
appropriate path.
If the file already exists the {\tt OVERWRITE} and
{\tt APPEND} buttons are enabled and the user must select
the appropriate action.
In all cases, the operation can be suspended without opening
a file with the {\tt ABORT} command.

{\bf Xzoom} always closes the current digitizing output file before
popping up the file name request, so that if an {\tt ABORT}
is issued the file remains closed and digitizing operations
are suspended.
\item[\fbox{\tt plotter}]
This command allows the user to specify a file containing
a list of overlay files to be included in graphic operations.
The file may previously exist, being created by non-{\bf xzoom} procedures.
It should contain only file names (and may have shell wild card characters)
or program {\bf plotter} pen mappings ({\tt -p}n{\tt :}m).
Newlines may be employed and a ``\verb@#@''
character will cause the character and the remainder of a line to be ignored.
The later feature allows the use of comments to describe overlay contents.

Again, the popup file selection window is used to enter the file name.
\item[\fbox{\tt edit pl}]
The file defined by the {\tt plotter} command may be edited or browsed
with a pop up window.
When finished the user executes the {\tt DONE} which will save
editing operations or if the user does not want not to
save the editing operations he or she may execute {\tt ABORT}.
If protections on the file does not allow user changes, the {\tt ABORT}
command is disabled and editing operations on the file are ignored
with the server's bell being sounded.
\end{description}
The next set of command buttons are associated with the graphic
display in the area at the bottom of the main {\bf xzoom} window
(a definition file must have been specified for these commands
to be activated):
\begin{description}
\item[\fbox{\tt (re)plot}]
This command causes plotting of the overlay files and is necessary
whenever the {\tt def}, {\tt plotter} or {\tt edit pl} commands have
been executed.
It may also be desirable to employ during digitizing operations
so that a true view of digitized events may be displayed.

Note that {\bf xzoom} performs graphics into a pixel map and
not directly onto the viewable screen.
After so many operations the viewable screen is updated and
thus causes a periodic burst of graphics on the visible screen.
The method is a compromise of inefficiently updating the screen
with every vector and delaying the display until the pixel map
is completed which may take a sufficient period of time to cause
concern to the user that the program is properly functioning.
\item[\fbox{\tt zoom}]
After selecting this command position the cursor at one corner
of the area to be expanded in the graphics region and click
mouse button 1.
Moving the cursor will show a box with one corner
anchored at the initial position.
Adjust the box to enclose the zoom region and press button
1 a second time and
{\bf xzoom} will automatically replot the graphics based on the
defined window.
Pressing either mouse button 2 or 3 will cancel the command.
The user may repeatatively increase zoom magnification several times.
\item[\fbox{\tt dezoom}]
If the graphic has been zoomed the previous zoom level may
be restored by this command.
The graphic is automatically replotted.
\item[\fbox{\tt pan}]
If the graphic has been zoomed the zoom window may be shifted
with this command.
First position the cursor on a point that is to be moved within
the limits of the current display and press button 1 on the mouse.
Move the cursor to where this point should be plotted on the
new display and press button 1 again.
Again, the graphic is automatically replotted.
To cancel the operation, press mouse button 2 or 3.
\item[\fbox{\tt home}]
This command is used
to return to the original full view, non-zoomed graphic which
will automatically be replotted.
\end{description}
The following commands allow digitizing operations provided
a {\tt digit} file has been specified:
\begin{description}
\item[\fbox{\tt line}]
After selecting this command position the cursor at line
node points in the graphic and press mouse button 1.
To cause the start of a new line segment press mouse button 2.
Mouse button 3 terminates the operation and reactivates
the basic command buttons.
\item[\fbox{\tt poly}]
This command is similar to {\tt line} except that the last
digitized point is connected to the first point of the line
segment.
\item[\fbox{\tt label}]
The contents of the text window above the graphics area along
with the coordinates determined when press mouse button 1 are
output.
By moving the cursor into the text window the text can be
edited at anytime during the {\tt label} operation.
Either mouse button 2 or 3 stops this operation.
\item[\fbox{\tt < label}]
Similar to the {\tt label} operation except that after selecting
the base point the user moves the mouse around to determine
the angle at which the label is to be printed and presses
mouse button 1 a second time.
Either mouse button 2 or 3 stops this operation.
\item[\fbox{\tt mark}]
This command only outputs the coordinates at the current
cursor positon when mouse button 1 is pressed.
No graphic operations are associated with this command.
Either mouse button 2 or 3 stops this operation.
\item[\fbox{\tt text out}]
The current contents of the text window are output
to the {\tt digit} file and thus
allows {\bf legend} control information to be entered
such as fonts and character sizes for label commands.
\end{description}

Commands which need file names
({\tt def}, {\tt cwd}, {\tt digit} and {\tt plotter})
cause a popup window for
entering the name required.
File names may use the {\bf csh} convention of prefixing
a tilde character to the path to indicate the user's or
other registered user's path to be prefixed to the
remaining name text.
Within this window other commands are enabled:
\begin{description}
\item[\fbox{\tt ENTER}]
Once the file name in the dialog window has been edited
to user's requirements entering the name is performed
by this command button.
\item[\fbox{\tt OVERWRITE}]
If the file is to be used for output and it
already exists when the {\tt ENTER}
command is pressed then this and the following commands
are enabled.
Selection of this command will cause the new data to
overwrite the existing file.
\item[\fbox{\tt APPEND}]
Selection of this command will
cause new digitized data to be appended to selected file.
\item[\fbox{\tt ABORT}]
If the user does not want to make any changes this command
terminates the process.
\end{description}

In the screen area just above the graphics the numbered
label buttons are commands which allow the user to select
the mechanical (or color) pen number to be output with
line, polynomial and label digitizing.
These can be selected at anytime and the
annotation box in the same section has the same foreground
and background color of the currently selected value.
\begin{figure*}[t]
\vspace{4.8in}
\special{psfile="`zcat full.ps"}
\caption{{\bf Xzoom} display of example overlay set.}
\label{full}
\end{figure*}

Figure~\ref{full} shows a plot of the example data set
distributed with {\bf xzoom} which was performed with a
{\tt cwd} to the example directory, {\tt def} command
using the default file name followed by {\tt (re)plot}.
This example also shows reference information in the
comment area giving plot {\tt size} (in centimeters) in terms of
the size intended for final hard copy, {\tt off}set of the lower
left hand corner of the screen display from the orgin of
the basic plot corner.
On the second line the depth of zoom level {\tt Zm lev:} and
approximate scale fraction of the intended
hard copy size to the actual display screen size.

\section{Digit file}
When performing digitizing operations with {\bf xzoom}
the file specified by the {\tt digit} command is
used for containing the {\bf legend} program control data
generated by the digitizing operations.
Anytime a {\tt (re)plot} command is issued this file
is processed by {\bf legend} creating a temporary overlay
file which is passed to program {\bf plotter}.

When performing either of the labeling commands the
a {\tt (re)plot} operation will not show either symbols
or characters without speccification of font and character size.
When a new digit file is openned or an old file is overwritten
{\bf xzoom} will prepend the following {\bf legend} control data
to the file:
\begin{verbatim}
-f -   # default font
-s .3  # size
-l .5  # spacing for multi-line text
\end{verbatim}
For other initializations, files containing commands
could be created and input to the text screen with
the Meta-{\tt i} editor command.

The resultant file created by the digitizing operations is often
post edited with finishing, cosmetic control information added.
Counterclockwise {\tt poly} operations are also often converted to {\bf vector}
masking control.
\section{Plotter file}
The {\tt plotter} file is used to create the program {\bf plotter}
execution script by appending its contents to the other
control information generated by {\bf xzoom}.
Except when being edited by the {\tt edit pl} command it
always resides on disk and can be edited from another window
with the user's favorite text editor.

Since the {\tt plotter} file is going to be processed by
a shell program usage of wild card characters for file
names is allowed.
But usage of other shell control characters should be done
with care and utmost understanding of the consequences.

For complex graphic operations which use a large number of
overlay files, users may find it useful to create several
of these {\tt plotter} files which select subsets suitable
for particular digiting or graphic operations.
They can of course be readily created and edited outside of {\bf xzoom}
with various {\sc Unix} utilities.

To overcome the case where the file is write protected
the user may execute the {\tt plotter} command
and create a personal plotter file and then initialize it
with {\tt edit pl} by using the Meta{\tt -i} editor command
to read the contents of the write protected plotter file.
If the current directory is also write protected the user must
locate the personal file in a writable directory.
\input keys
\input XZoom
\end{document}
