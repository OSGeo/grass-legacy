\documentstyle[transfig]{article}
\title{Image Sampling Methods\\ {\large GRASS Tutorial on {\tt s.sample}}}
\author{James Darrell McCauley}
\date{04 January 1993}
\def\fphi#1{\phi\left(#1\right)}
\begingroup\makeatletter
% extract first six characters in \fmtname
\def\x#1#2#3#4#5#6#7\relax{\def\x{#1#2#3#4#5#6}}
\expandafter\x\fmtname xxxxxx\relax \def\y{splain}
\ifx\x\y   % LaTeX or SliTeX?
\gdef\SetFigFont#1#2#3{%
  \ifnum #1<17 \tiny\else \ifnum #1<20 \small\else
  \ifnum #1<24 \normalsize\else \ifnum #1<29 \large\else
  \ifnum #1<34 \Large\else \ifnum #1<41 \LARGE\else
     \huge\fi\fi\fi\fi\fi\fi
  \csname #3\endcsname}
\else
\gdef\SetFigFont#1#2#3{\begingroup
  \count@#1\relax \ifnum 25<\count@ \count@25 \fi
  \def\x{\endgroup\@setsize\SetFigFont{#2pt}}%
  \expandafter\x
    \csname \romannumeral\the\count@ pt\expandafter\endcsname
    \csname @\romannumeral\the\count@ pt\endcsname
  \csname #3\endcsname}
\fi
\endgroup
\begin{document}
\bibliographystyle{plain}
\maketitle

\begin{abstract}
The sampling methods used by the GRASS program {\tt s.sample}~\cite{s.sample}
are presented. These are: nearest neighbor sampling, 
bilinear interpolation, and cubic convolution
interpolation.
\end{abstract}

Interpolation software often writes output as images rather
than punctual form. With a defined resolution in the {\em x-y\/}
plane, picture elements, known as {\em pixels\/} or {\em cells},
assume a value for a small rectangular region. This value is
the result of interpolating at the center of the pixel.

This output representation poses problems for cross-validation.
Data for cross validation, like the data used for interpolation,
are not gridded but rather scattered. Their locations in the
{\em x-y\/} plane will not necessarily coincide with the
centers of pixels. Therefore, a {\em z}-value at the
exact location of the point used for cross-validation
may not be obtainable from the output image. However, there
are ways of approximating values. Three such methods are
nearest neighbor, bilinear interpolation, and cubic convolution
interpolation~\cite{richards93}. These are implemented
in the GRASS program {\tt s.sample}~\cite{s.sample} and discussed below.

\begin{figure}
\begin{center}
\par
\setlength{\unitlength}{0.007500in}%
\begin{picture}(267,130)(3,690)
\thicklines
\put( 80,820){\line( 0,-1){120}}
\put(160,820){\line( 0,-1){120}}
\put( 40,800){\line( 1, 0){160}}
\put( 40,720){\line( 1, 0){160}}
\multiput(130,820)(0.00000,-7.74194){16}{\line( 0,-1){  3.871}}
\put( 80,710){\vector(-1, 0){  0}}
\put( 80,710){\vector( 1, 0){ 50}}
\put(180,800){\vector( 0, 1){  0}}
\put(180,800){\vector( 0,-1){ 50}}
\multiput( 40,750)(7.80488,0.00000){21}{\line( 1, 0){  3.902}}
\put( 70,805){\makebox(0,0)[rb]{\smash{\SetFigFont{7}{8.4}{rm}$\fphi{i,j}$}}}
\put( 70,700){\makebox(0,0)[rb]{\smash{\SetFigFont{7}{8.4}{rm}$\fphi{i+1,j}$}}}
\put(170,805){\makebox(0,0)[lb]{\smash{\SetFigFont{7}{8.4}{rm}$\fphi{i,j+1}$}}}
\put(105,690){\makebox(0,0)[b]{\smash{\SetFigFont{7}{8.4}{rm}$j'$}}}
\put(190,770){\makebox(0,0)[lb]{\smash{\SetFigFont{7}{8.4}{rm}$i'$}}}
\put(270,700){\makebox(0,0)[lb]{\smash{\SetFigFont{7}{8.4}{rm}This value used.}}}
\put(260,700){\makebox(0,0)[rb]{\smash{\SetFigFont{7}{8.4}{rm}$\fphi{i+1,j+1}$}}}
\put(260,805){\makebox(0,0)[lb]{\smash{\SetFigFont{7}{8.4}{rm}Pixel values}}}
\end{picture}
\centerline{(a) nearest neighbor}
\vspace{\baselineskip}
\par
\setlength{\unitlength}{0.007500in}%
\begin{picture}(461,181)(8,635)
\thicklines
\put(160,755){\line( 0, 1){ 60}}
\put(320,775){\line( 0,-1){ 20}}
\multiput(180,710)(31.48853,31.48853){3}{\line( 1, 1){ 17.023}}
\put(260,790){\line( 0,-1){ 35}}
\put(240,735){\line( 0, 1){ 35}}
\multiput( 80,735)(45.54824,-11.38706){4}{\line( 4,-1){ 23.355}}
\multiput(160,816)(45.54824,-11.38706){4}{\line( 4,-1){ 23.355}}
\put( 80,735){\line( 0,-1){ 60}}
\put(240,695){\line( 0,-1){ 20}}
\put(180,710){\line( 0,-1){ 35}}
\put( 40,635){\line( 1, 1){160}}
\put(200,635){\line( 1, 1){160}}
\put( 40,675){\line( 1, 0){240}}
\put(120,755){\line( 1, 0){240}}
\multiput(170,665)(30.99235,30.99235){4}{\line( 1, 1){ 17.023}}
\multiput(100,735)(7.86885,0.00000){31}{\line( 1, 0){  3.934}}
\put(240,750){\line( 2,-1){ 60}}
\put(155,785){\makebox(0,0)[rb]{\smash{\SetFigFont{7}{8.4}{rm}$\fphi{i,j}$}}}
\put( 75,700){\makebox(0,0)[rb]{\smash{\SetFigFont{7}{8.4}{rm}$\fphi{i+1,j}$}}}
\put(265,680){\makebox(0,0)[lb]{\smash{\SetFigFont{7}{8.4}{rm}$\fphi{i+1,j+1}$}}}
\put(340,760){\makebox(0,0)[lb]{\smash{\SetFigFont{7}{8.4}{rm}$\fphi{i,j+1}$}}}
\put(270,800){\makebox(0,0)[b]{\smash{\SetFigFont{7}{8.4}{rm}$\fphi{i,j'}$}}}
\put(160,645){\makebox(0,0)[b]{\smash{\SetFigFont{7}{8.4}{rm}$\fphi{i+1,j'}$}}}
\put(370,715){\makebox(0,0)[lb]{\smash{\SetFigFont{7}{8.4}{rm}Pixel value required}}}
\put(360,715){\makebox(0,0)[rb]{\smash{\SetFigFont{7}{8.4}{rm}$\fphi{i',j'}$}}}
\end{picture}
\centerline{(b) bilinear interpolation}
\vspace{\baselineskip}
\par
\setlength{\unitlength}{0.007500in}%
\begin{picture}(474,302)(21,510)
\thicklines
\multiput(185,725)(-0.66667,-0.66667){16}{\makebox(0.7407,1.1111){\SetFigFont{7}{8.4}{rm}.}}
\multiput(175,725)(0.66667,-0.66667){16}{\makebox(0.7407,1.1111){\SetFigFont{7}{8.4}{rm}.}}
\multiput(185,665)(-0.66667,-0.66667){16}{\makebox(0.7407,1.1111){\SetFigFont{7}{8.4}{rm}.}}
\multiput(175,665)(0.66667,-0.66667){16}{\makebox(0.7407,1.1111){\SetFigFont{7}{8.4}{rm}.}}
\multiput(185,605)(-0.66667,-0.66667){16}{\makebox(0.7407,1.1111){\SetFigFont{7}{8.4}{rm}.}}
\multiput(175,605)(0.66667,-0.66667){16}{\makebox(0.7407,1.1111){\SetFigFont{7}{8.4}{rm}.}}
\multiput(185,545)(-0.66667,-0.66667){16}{\makebox(0.7407,1.1111){\SetFigFont{7}{8.4}{rm}.}}
\multiput(175,545)(0.66667,-0.66667){16}{\makebox(0.7407,1.1111){\SetFigFont{7}{8.4}{rm}.}}
\put( 80,740){\line( 0,-1){220}}
\put(200,740){\line( 0,-1){220}}
\put(260,740){\line( 0,-1){220}}
\put(140,740){\line( 0,-1){220}}
\put( 60,720){\line( 1, 0){220}}
\put( 60,660){\line( 1, 0){220}}
\put( 60,540){\line( 1, 0){220}}
\multiput(180,740)(0.00000,-8.00000){28}{\line( 0,-1){  4.000}}
\put(220,660){\vector( 0, 1){  0}}
\put(220,660){\vector( 0,-1){ 40}}
\put( 60,600){\line( 1, 0){220}}
\multiput( 60,620)(8.00000,0.00000){28}{\line( 1, 0){  4.000}}
\put(140,525){\vector(-1, 0){  0}}
\put(140,525){\vector( 1, 0){ 40}}
\put(340,720){\vector(-1, 0){ 40}}
\put(340,660){\vector(-1, 0){ 40}}
\put(340,600){\vector(-1, 0){ 40}}
\put(340,540){\vector(-1, 0){ 40}}
\put(340,720){\line( 2,-3){ 60}}
\put(340,660){\line( 2,-1){ 60}}
\put(340,600){\line( 2, 1){ 60}}
\put(340,540){\line( 2, 3){ 60}}
\put(180,790){\vector( 0,-1){ 40}}
\put(410,617){\makebox(0,0)[lb]{\smash{\SetFigFont{7}{8.4}{rm}interpolations}}}
\put(410,635){\makebox(0,0)[lb]{\smash{\SetFigFont{7}{8.4}{rm}Cubic polynomial}}}
\put(270,520){\makebox(0,0)[lb]{\smash{\SetFigFont{7}{8.4}{rm}$\fphi{i+3,j+3}$}}}
\put(225,635){\makebox(0,0)[lb]{\smash{\SetFigFont{7}{8.4}{rm}$i'$}}}
\put(160,510){\makebox(0,0)[b]{\smash{\SetFigFont{7}{8.4}{rm}$j'$}}}
\put(180,800){\makebox(0,0)[b]{\smash{\SetFigFont{7}{8.4}{rm}Cubic polynomial interpolation}}}
\put( 75,730){\makebox(0,0)[rb]{\smash{\SetFigFont{7}{8.4}{rm}$\fphi{i,j}$}}}
\end{picture}
\centerline{(c) cubic convolution interpolation}
\end{center}

\caption{Image sampling using (a) nearest neighbor, (b) bilinear interpolation,
and (c) cubic convolution interpolation; pixel centers are at crossings, 
unknown point is located at $(i',j')$, grid spacing of unity is assumed.
Adapted from Richards~\protect\cite{richards93}.}
\label{fig:sampling}
\end{figure}

{\em Nearest neighbor\/} sampling is probably the simplest approach
to approximating values.
In this method, the value of the pixel whose center is nearest to 
the cross-validation point is used. Figure~\ref{fig:sampling}
shows this graphically.

{\em Bilinear interpolation\/} is another method. It uses a 
pixel values ($\phi$) of a
$2\times2\/$ neighborhood of pixels to perform three linear
interpolations (see fig.~\ref{fig:sampling}). First, two
interpolations are performed, one along the row of pixels above the
point and one along the row of pixels below the point, yielding
the interpolants $\fphi{i, j'}\/$ and $\fphi{i+1, j'}\/$ given
by:

\begin{eqnarray}
  \fphi{i, j'} & = & j' \fphi{i,j+1} + \left(1-j'\right)\fphi{i,j}\\
\fphi{i+1, j'} & = & j' \fphi{i+1,j+1} + \left(1-j'\right)\fphi{i+1,j}
\end{eqnarray}

\noindent These equations assume a grid spacing of unity.
The position of the cross validation point, $\left(i',j'\right),$ 
is measured relative to $\left(i,j\right).$  The final
step is to interpolate linearly over the two interpolants
$\fphi{i, j'}\/$ and $\fphi{i+1, j'},$ yielding:

\begin{eqnarray}
  \fphi{i',j'} & = & \left(1-i'\right)\left[j'\fphi{i,j+1}+\left(1-j'\right)
                                           \fphi{i,j}\right]\\ \nonumber
                &   & + i'\left[j'\fphi{i+1,j+1}+\left(1-j'\right)
                             \fphi{i+1,j}\right]
\end{eqnarray}
where $\fphi{i',j'}\/$ is the estimated value of the cross
validation point.

{\em Cubic convolution interpolation\/} is another method of finding
a value at $\left(i',j'\right).$ It uses a $4\times4\/$ neighborhood 
of pixels. The algorithm that is used to perform this
operation follows the following equation:
\begin{eqnarray}
\fphi{i,j'} & = & j'\left\{j'\left(j'\left[\fphi{i,j+3}-\fphi{i,j+2}
                  + \fphi{i,j+1}-\fphi{i,j}\right]\right.\right.\\
                  \nonumber
            &   & + \left.\left.\left[\fphi{i,j+2}-\fphi{i,j+3}-2\fphi{i,j+1}
                  + 2\fphi{i,j}\right]\right)\right.\\
                  \nonumber
            &   & + \left.\left[\fphi{i,j+2} - \fphi{i,j}\right]\right\}\\
                  \nonumber
            &   & + \fphi{i,j+1}
\end{eqnarray}
This expression is evaluated for each row of the $4\times4\/$ neighborhood
shown in figure~\ref{fig:sampling} to yield the four interpolants
$\fphi{i,j'}$, $\fphi{i+1,j'}$, $\fphi{i+2,j'}$, $\fphi{i+3,j'}$.
These are interpolated parallel to pixel columns according to:
\begin{eqnarray}
\fphi{i',j'} & = & i'\left\{i'\left(i'\left[\fphi{i+3,j'}-\fphi{i+2,j'}
                  + \fphi{i+1,j'}-\fphi{i,j'}\right]\right.\right.\\
                  \nonumber
            &   & + \left.\left.\left[\fphi{i+2,j'}-\fphi{i+3,j'}-2\fphi{i+1,j'}
                  + 2\fphi{i,j'}\right]\right)\right.\\
                  \nonumber
            &   & + \left.\left[\fphi{i+2,j'} - \fphi{i,j'}\right]\right\}\\
                  \nonumber
            &   & + \fphi{i+1,j'}
\end{eqnarray}

Nearest neighbor sampling is the simplest method and
retains original pixel values. This may be desirable if the
purpose for sampling is to perform
classifications~\cite{richards93}.  Bilinear interpolation
is very intuitive and requires few operations.  Cubic
convolution also requires few operations but requires 4
times as many pixels to be stored. The size of the
neighborhood considered for sampling becomes important when
considering locations near edges of image boundaries.

\begin{thebibliography}{1}
\bibitem{s.sample}
James~Darrell McCauley.
\newblock {\em s.sample: software to sample raster maps. {GRASS} 4.2 Reference
  Manual}.
\newblock U.S. Army Corps of Engineers, Construction Engineering Research
  Laboratories, Champaign, Illinois, to appear 1994.
 
\bibitem{richards93}
John~A. Richards.
\newblock {\em Remote Sensing Digital Image Analysis}.
\newblock Springer-Verlag, Berlin, 2nd edition, 1993.
 
\end{thebibliography}

\end{document}
